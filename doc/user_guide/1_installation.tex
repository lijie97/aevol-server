\chapter{Installation}
\label{chap:install}

% Apple changes its OS name every now and then...
% Mac OS X, OS X will soon be macOS
\newcommand{\macos}{OS X}

\vspace{5mm}

\aevol{} can run on Linux and on \macos{}.

\section{Linux users}
\label{sec:linux}

\subsection{Pre-built packages}
\aevol{} 4.4 is available as a deb package in all the
repositories. So, you can \verb?apt install aevol? provided the 4.4
version fits your needs. \aevol{} 5 should also hit the repositories
soon.

\subsection{Installation from Source}

Installation commands are given for Debian-like systems (\verb?apt?)
as well as Fedora-like systems (\verb?dnf?).

\subsubsection{Required Dependencies}
\begin{myList} 
\item \textbf{Build Tools}.\\
  \verb?apt install build-essential? or \verb?dnf install gcc-c++?.
\item \textbf{Compression library}.
  \aevol{} compresses most of the data it generates.\\
  \verb?apt install zlib1g-dev? or \verb?dnf install zlib-devel?.
\item \textbf{Boost library}.
  \aevol{} also relies on the Boost Filesystem library.\\
  \verb?apt install libboost-filesystem-dev? or \verb?dnf install boost-devel?.
\end{myList}

\subsubsection{Optional Dependencies}
\begin{myList} 
\item \textbf{X libraries}.  \aevol{} uses the X11 library for the
  graphical outputs.\\
  \verb?apt install libx11-dev? or \verb?dnf install libX11-devel?.

Note, however, that \aevol{} can be compiled without graphical
outputs, and hence no need for X libraries, by typing
\verb?./configure --without-x?  instead of \verb?./configure? (see
installation instructions below for more information). This option is
useful if you want to run \aevol{} on a computer cluster, for example.
\end{myList}

\subsubsection{Installation Instructions}
Download the latest release of \aevol{} at
\url{http://aevol.fr/download/} and save it to a directory of your
choice.  Open a terminal and use the \verb+cd+ command to navigate to
this directory.  Then follow the steps below to extract the files and
build the executables:

\begin{verbatim}
	tar zxf aevol-VERSION.tar.gz
	cd aevol-VERSION
	./configure
	make
\end{verbatim}

If you have administration privileges, you can finally make the
\aevol{} programs available to all users on the computer by typing:
\begin{verbatim}
	sudo make install
\end{verbatim}

If you don't have administration privileges, you may still install
\aevol{} ``locally'' by doing the following:
\begin{verbatim}
	./configure --prefix=/install/path
	make
	make install
\end{verbatim}
where \verb?/install/path? is a directory where you have write
permission. Don't forget to add \verb?/install/path? to your
\verb?PATH? environment variable.



\section{Mac users}
\label{sec:mac}

\subsection{Pre-built packages}
This option is not available yet for Mac users.


\subsection{Installation from Source}
\subsubsection{Required Dependencies}
\begin{myList} 
\item \textbf{C++ command-line compiler}. Mac users need a
  command-line C++ compiler like \verb?g++? or \verb?clang?
  installed.

  There are three options:
  \begin{description}
  \item [App Store] If your computer runs OS X El Capitan (10.11), you
    can first install XCode (freely downloadable from the App Store),
    then start XCode and install the Command Line Tools package from
    the menu XCode / Preferences / Downloads / tab `` Components''.
  \item [Standalone Command Line Tools] Alternatively, but still
    provided your computer runs OS X El Capitan (10.11), you can
    install the Command Line Tools package for Xcode (without
    installing Xcode itself) by downloading it from Apple's developer
    site. A free registration required there, then just search for
    ``Command Line Tools''.
  \item [Homebrew] For all versions of \macos{}, another way is to
    install Homebrew to get the appropriate packages. To install
    Homebrew, you have to open a terminal and type (or copy/paste,
    more likely) the following command:
  \begin{verbatim}
    /usr/bin/ruby -e "$(curl -fsSL https://raw.githubusercontent.com/
        Homebrew/install/master/install)"
  \end{verbatim}

  Then you can install the appropriate dependencies with:
  \begin{verbatim}
    brew install boost clang-omp
  \end{verbatim}
  \end{description}
  
\item \textbf{Compression library}. \aevol{} compresses most of the
  data it uses using the zlib1g library. This library is already
  included as part of \macos{} so there is no need to install it.
\end{myList}
\subsubsection{Optional Dependencies}
\begin{myList} 
\item \textbf{X libraries}. For the graphical outputs, Mac users
  should also have X11 installed. X11 is not included with \macos{},
  but X11 server and client libraries for \macos{} are available from the
  XQuartz project. You will need
  to log out and log in after the installation to have X11 properly
  setup. Note, however, that \aevol{} can be compiled without
  graphical outputs, and hence no need for X libraries, by typing
  \verb?./configure --without-x?  instead of \verb?./configure?  (see
  below). This option is useful if you want to run \aevol{} on a
  computer cluster, for example.
  
  You can either download XQuartz manually from
  (\url{http://xquartz.macosforge.org}) or, if you previously chose
  the Homebrew way, you may just
  \verb?brew install Caskroom/cask/xquartz?. In either case, don't
  forget to re-log for proper install to be completed.

\end{myList}

\subsubsection{Installation Instructions}
Download the latest release of \aevol{} at
\url{http://aevol.fr/download/} and save it to a directory of your
choice. Open a terminal and use the \verb+cd+ command to navigate to
this directory. Then follow the steps below to extract the files and
build the executables:

\begin{verbatim}
	tar zxf aevol-VERSION.tar.gz
	cd aevol-VERSION
	./configure
	make
\end{verbatim}

If you have administration privileges, you can finally make the
\aevol{} programs available to all users on the computer by typing:
\begin{verbatim}
	sudo make install
\end{verbatim}

If you don't have administration privileges, you may still install
\aevol{} ``locally'' by doing the following:
\begin{verbatim}
	./configure --prefix=/install/path
	make
	make install
\end{verbatim}
where \verb?/install/path? is a directory where you have write
permission. Don't forget to add \verb?/install/path? to your
\verb?PATH? environment variable.


\clearemptydoublepage






