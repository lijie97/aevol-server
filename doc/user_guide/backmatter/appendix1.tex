\chapternonum{Appendix 1: Man Pages}
\label{app:manpages}
\addcontentsline{toc}{chapter}{Appendix 1: Man Pages}


\section{aevol\_create}
\label{sec:aevol-create}

\subsection*{Synopsis}
\begin{verbatim}
    aevol_create -h or --help
   	aevol_create -V or --version
   	aevol_create [-f param_file]
\end{verbatim}

\subsection*{Description}
Create an experiment with setup as specified in param\_file.

\subsection*{Options}
\begin{verbatim}
    -h, --help
        print this help, then exit

   	-V, --version
        print version number, then exit

   	-f, --file param_file
        specify parameter file (default: param.in)
\end{verbatim}

\newpage


\section{aevol\_run}
\label{sec:aevol-run}

\subsection*{Synopsis}
\begin{verbatim}
    aevol_run -h | --help
    aevol_run -V | --version
    aevol_run [-r GENER] [-n NB_GENER] [-tvwx]
\end{verbatim}

\subsection*{Description}
Run an \aevol{} simulation.

\subsection*{Options}
\begin{verbatim}
	-h, --help
    	print this help, then exit

	-V, --version
	    print version number, then exit

	-r, --resume GENER
	    specify generation to resume simulation at (default 0)

	-n, --nbgener NB_GENER
	    specify number of generations to be run (default 1000)

	-t, --text
	    use text files instead of binary files when possible

	-v, --verbose
	    be verbose
\end{verbatim}

\newpage


\section{aevol\_propagate}
\label{sec:aevol-propagate}

\subsection*{Synopsis}
\begin{verbatim}
   aevol_propagate -h or --help
   aevol_propagate -V or --version
   aevol_propagate [-v] [-g GENER] [-i INDIR] [-o OUTDIR] [-S GENERALSEED]
   aevol_propagate [-v] [-g GENER] [-i INDIR] [-o OUTDIR] [-s SELSEED] 
             [-m MUTSEED] [-t STOCHSEED] [-e ENVVARSEED] [-n ENVNOISESEED]
\end{verbatim}

\subsection*{Description}
\verb?aevol_propagate? creates a fresh copy of the experiment as it was at the given generation. The generation number of the copy will be reset to 0. If you use \verb?aevol_propagate? repeatedly to initialize several simulations, you should specify a different seed for each simulation, otherwise all simulations  will  yield  exactly  the same results. You can use the option -S to do so. In this case, the random drawings will be different for all random processes enabled in your  simulations  (mutations, stochastic gene expression, selection, migration, environmental variation, environmental noise). Alternatively, to change the random drawings for specific  random  processes only, do not use -S but the options -m, -s, -t, -e, -n (see below).

\subsection*{Options}
\begin{verbatim}
	-h, --help
	  	print this help, then exit

	-V, --version
	    print version number, then exit

	-v, --verbose
	    be verbose

	-g, --gener GENER
	    specify generation number (default: that contained in file 
	    last_gener.txt, if any)

	-i, --in INDIR
	    specify input directory (default ".")

	-o, --out OUTDIR
	    specify output directory (default "./output")

	-S, --general-seed GENERALSEED
	    specify  an  integer  to  be  used  as  a seed for random 
	    numbers. If you use aevol_propagate repeatedly to initialize
	    several simulations, you should specify a  different seed
	    for each simulation, otherwise all simulations will yield 
	    exactly the same results. If you specify this general seed,
	    random drawings will be different for all random processes 
	    enabled in your simulations (mutations, stochastic gene 
	    expression, selection, migration, environmental variation, 
	    environmental noise). To change the random drawings for a 
	    specific random process only, do not use -S but the options 
	    below.

	-s, --sel-seed SELSEED
	    specify  an  integer as a seed for random numbers needed 
	    for selection and migration (if spatial structure is enabled).

	-m, --mut-seed MUTSEED
	    specify an integer as a seed for random numbers needed for 
	    mutations.

	-t, --stoch-seed STOCHSEED
	    specify an integer as a seed for random numbers needed for
	    stochastic gene expression.

	-e, --env-var-seed ENVVARSEED
	    specify an integer as a seed for random numbers needed for 
	    environmental variation.

	-n, --env-noise-seed ENVNOISESEED
	    specify an integer as a seed for random numbers needed for 
	    environmental noise.

\end{verbatim}

\newpage


\section{aevol\_modify}
\label{sec:aevol-modify}

\subsection*{Synopsis}
\begin{verbatim}
    aevol_modify -h or --help
    aevol_modify -V or --version
    aevol_modify -g GENER [-f param_file]
\end{verbatim}

\subsection*{Description}
Modify an experiment as specified in param\_file.

\subsection*{Options}
\begin{verbatim}
	-h, --help
	    print this help, then exit

	-V, --version
	    print version number, then exit

	-g, --gener GENER
	    specify generation number

	-f, --file param_file
	    specify parameter file (default: param.in)
\end{verbatim}


\clearemptydoublepage