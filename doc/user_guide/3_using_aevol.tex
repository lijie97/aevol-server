\chapter{Using Aevol}
\label{chap:using-aevol}


\vspace{5mm}

\aevol{} is made up of 4 main tools (\verb?aevol_create?, \verb?aevol_run?, \verb?aevol_propagate? and \verb?aevol_modify? -- man pages provided in appendix 1) and a set of post-treatment tools (prefixed by \verb?aevol_misc_?).

If you have tried one of the examples provided (see \ref{chap:first-runs}), you may have noticed that a call to \verb?aevol_create? produces a whole set of files organized in different directories. Everything in \aevol{} relies on this file organization where all the data for your experiment is stored: organisms in the \verb?population? directory, the task they are selected for in \verb?environment?, the experimental setup in \verb?exp_setup? and so on. It is not recommended to manually modify these files since this may cause some inconsistency leading to undefined behaviour. Besides, most of these files are compressed.

Once created, an experiment can either be run, propagated or modified.
Propagating an experiment means creating a fresh copy of it (setting the current generation number to 0). It can be though of as a bacterial colony picking save that you can make as many strictly identical copies of the whole original population as you wish.

The \verb?aevol_modify? tool virtually allows for the modification of any parameter of the experiment, including manipulations of the population or of individual organisms (\emph{e.g.} ``I want the population to be filled with clones of the organism having the longest genome'' or ``I want a random subset of organisms to be switched to super mutators''). To date, only the most common experiment modifications have been implemented but feel free to ask for more (\verb?aevol-feat-request@lists.gforge.liris.cnrs.fr?).

The typical workflow of a campain of experiments using aevol is the following:

\begin{enumerate}
\item Create a Wild Type strain
\begin{verbatim}
	aevol_create
	aevol_run
\end{verbatim}

\item Create many replicates of the wild type thus created
\begin{verbatim}
	aevol_propagate
\end{verbatim}

\item Modify some parameter(s) of your choice in your different replicates
\begin{verbatim}
	aevol_modify
\end{verbatim}

\item Let the colonies evolve for some time
\begin{verbatim}
	aevol_run
\end{verbatim}

\item Analyse the outcome
\end{enumerate}


\clearemptydoublepage






