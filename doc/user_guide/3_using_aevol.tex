\chapter{Tutorial: Using \aevol{}}
\label{chap:using-aevol}


\vspace{5mm}

\section{Introduction}

\aevol{} is made up of 4 main tools (\verb?aevol_create?, \verb?aevol_run?, \verb?aevol_propagate? and \verb?aevol_modify? -- man pages provided in appendix 1) and a set of post-treatment tools (prefixed by \verb?aevol_misc_?).

%If you have tried one of the examples provided (see \ref{chap:first-runs}), you may have noticed that a call to \verb?aevol_create? produces a whole set of files organized in different directories.
Everything in \aevol{} relies on an ad-hoc file organization where all the data for an experiment is stored: organisms in the \verb?population? directory, the task they are selected for in \verb?environment?, the experimental setup in \verb?exp_setup? and so on. It is not recommended to manually modify these files since this may cause some inconsistency leading to undefined behaviour. Besides, most of these files are compressed.

Once created, an experiment can either be run, propagated or modified.

\vspace{-7mm}
\paragraph{Running}an experiment simply means simulate evolution for a given number of generations.

\vspace{-7mm}
\paragraph{Propagating}an experiment means creating a fresh copy of it (setting the current generation number to 0). It can be though of as a bacterial colony picking save that you can make as many strictly identical copies of the whole original population as you wish.

\vspace{-7mm}
\paragraph{Modifying}an experiment actually means modifying some of its parameters. The \verb?aevol_modify? tool virtually allows for the modification of any parameter of the experiment, including manipulations of the population or of individual organisms (\emph{e.g.} ``I want the population to be filled with clones of the organism having the longest genome'' or ``I want a random subset of organisms to be switched to super mutators''). To date, only the most common experiment modifications have been implemented but feel free to ask for more (\verb?aevol-feat-request@lists.gforge.liris.cnrs.fr?).



\aevol{} comes along with a set of simple but representative examples. Following these examples is probably the best way to get going with \aevol{} and have a quick overview of the possibilities it offers. In any case, keep in mind that you can always get help by typing \verb?man aevol_cmd? (only available for the 4 main commands) or \verb?aevol_cmd -h? (available for all the commands).

Most examples are showcases for different features of the model such as spatially structured populations, plasmids and horizontal transfer. They can all be run with the same very simple commands. Simply follow the instructions from section \ref{sect:basic_examples}.
The \verb?workflow? example proposes a typical \emph{``experiments on a previously generated wild-type''} workflow. It will lead you through the whole experimental process, including a sample of possible post-treatments you can use to analyse the outcome of your different simulations.


\section{Basic examples}
\label{sect:basic_examples}

To run all but the \verb?workflow? examples, simply follow the following steps:

\begin{enumerate}
\item Install \aevol{}, preferentially with graphics enabled (see chapter \ref{chap:install})
\item cd into the directory of the example (\emph{e.g.} \verb?examples/basic?)
\item
run \verb?aevol_create?
\item
run \verb?aevol_run_X11?
\item Have a look at the graphical outputs (Ctrl+Q to quit)
% \item [Optional] Explore the different stats created (in the \verb?stats? directory)
\end{enumerate}


\section{The \emph{workflow} example}
The workflow example provides an example of one of the many different workflows that can be used for experiments with \aevol{}.

The main idea underlying this workflow is to parallel wet lab experiments, which are conducted on evolved organisms.
To use already evolved organisms for \aevol{} experiments, one can either use an evolved genome provided by the community or evolve his own. This example describes the latter (more complete) case.


\subsection{\emph{Wild-Type} generation}
Generating a Wild-Type in \aevol{} is very easy, all you need is a parameter file describing the conditions in which it (the Wild-Type) should be created (population size, mutation rates, task to perform, ...).
However, have in mind that founding effects can influence the course of evolution, especially in the case of overconstrained evolution. It is recommended to use mild mutation and rearrangement rates and to let the environment vary over time to avoid overconstrained or overspecialized genomes.
Once your parameter file is ready, simply run the following commands (it is recommended you do that in a dedicated directory):

\begin{verbatim}
	aevol_create -f your_param_file
	aevol_run -n number_of_generations
\end{verbatim}

%$number\_of\_generations$ should be on the order of a few thousand...


\subsection{Experimental setup}
This is where the setup of the campain of experiments is done.
As it would be done in a wet lab experiment, different populations will be allowed to evolve in different conditions to compare the different outcomes. In this example we will let 6 populations evolve independently for 20,000 generations in conditions differing only by the spontaneous rates of chromosomal rearrangements. A ``real'' experiment would of course require several repetitions for each set of parameters, this is \emph{only} an example.

The \verb?aevol_propagate? tool provided in \aevol{} allows for an exact copy of the whole data structure required by \aevol{} with a reset of the current generation number to 0. Followed by a call to \verb?aevol_modify?, it allows us to set up very simply our example in the 2 following steps:

%\begin{enumerate}

\subsubsection{Propagate the experiment}
The \verb?aevol_propagate? tool allows for the creation of fresh copies of an experiment (as it was at a given time). The \verb?-i? option sets the input directory and the \verb?-o? option, the output directory. You must provide a distinct output directory for each of the experiments you wish to run. If the output directory does not exist, you will be prompted for it to be created.
\begin{verbatim}
	aevol_propagate -i wild_type -o mu_1e-6
	aevol_propagate -i wild_type -o mu_2e-6
	aevol_propagate -i wild_type -o mu_5e-6
	aevol_propagate -i wild_type -o mu_1e-5
	aevol_propagate -i wild_type -o mu_2e-5
	aevol_propagate -i wild_type -o mu_5e-5
\end{verbatim}

\subsubsection{Modify parameters to meet the experiment requirements}
For each of the propagated experiments, create a plain text file (\emph{e.g.} ``foo.in'') containing the parameters to be modified. Parameters that do not appear in this file will remain unchanged. The syntax is the same as for the parameter file used for \verb?aevol_create?. For example, to set all the rearrangement rates to $1\times10^{-6}$, the file ``foo.in'' will consist in the following 4 lines:
\begin{verbatim}
    DUPLICATION_RATE        1e-6
    DELETION_RATE           1e-6
    TRANSLOCATION_RATE      1e-6
    INVERSION_RATE          1e-6
\end{verbatim}
Then run the following:
\begin{verbatim}
	aevol_modify --gener 0 --file foo.in
\end{verbatim}

%\end{enumerate}


\subsection{Run the simulations}
Each of the propagated experiments can be run thus:
\begin{verbatim}
	aevol_run -n <number_of_generations>
\end{verbatim}
Of course, all the runs being completely independent, you can submit these tasks to a cluster of your choice to save time.


\subsection{Analyse the outcome}

In addition to the set a statistics files that are recorded in the \verb?stats? directory, \aevol{} includes a set of post-treatment tools to further analyse the outcome of your experiments, please refer to section \ref{sect:post-treatments}.


\section{Post-treatment Tools}
\label{sect:post-treatments}

In addition to the set a statistics files that are recorded in the \verb?stats? directory, \aevol{} includes a set of post-treatment tools to further analyse the outcome of your experiments.

Please note that these tools have only been tested on simple experimental setups and will fail with exotic ones.
However, in most cases, these problems can easily be remedied. Please do not hesitate to send us your request (aevol-feat-request@lists.gforge.liris.cnrs.fr).

\subsection{aevol\_misc\_view\_generation}
\label{sect:view-gener}
The \verb?view_generation? tool is probably the easiest and most straightforward tool provided with \aevol{}.
It allows one to visualize a generation using the exact same graphical outputs used in \verb?aevol_run?.
However, since it relies on graphics, it is only available when \aevol{} is compiled with x enabled (which is the default).

Usage: \verb?aevol_misc_view_generation -g generation_number?

\subsection{aevol\_misc\_create\_eps}
\label{sect:create-eps}

\subsection{aevol\_misc\_lineage}
\label{sect:lineage}
The \verb?lineage? tool allows for the reconstruction of the lineage of a given individual. It requires the phylogenetic tree to be issued during the evolutionnary run (see the \verb?TREE_MODE? parameter). Using this phylogenetic tree, it will produce a file containing the whole evolutionary history of any given individual, \emph{i.e.} for each of its ancestors, which organism in the previous generation it is an offspring of, and the list of mutations that occured during replication. This file will be named \emph{e.g.} \\\verb?lineage-b000000-e050000-i999-r1000.ae? which means we retraced the evolutionary history of the organism with rank $1,000$ (that had the index $999$) at generation $50,000$ and that its history was retraced all the way down to generation $0$.

Usage: \verb?aevol_misc_lineage [-i index | -r rank] [-b gener1] -e gener2?

\subsection{aevol\_misc\_ancstats}
\label{sect:ancstats}
The \verb?ancstats? tool issues the ``statistics'' for the lineage of a given individual (providing its lineage file, see section \ref{sect:lineage}). It will produce a set of files similar to those created in the \verb?stats? directory during the simulation but regarding the provided lineage instead of the best organism of each generation. These files are placed in the \verb?stats/ancstats? directory.

Usage: \verb?ae_misc_ancstats -f lineage_file?


\clearemptydoublepage






